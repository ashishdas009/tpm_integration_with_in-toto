\section{System Overview}
One of the key security goals of in-toto are artifact flow integrity and step authentication, which will be compromised if a functionary is compromised. The attacker on the specific functionary can arbitarily forge the link metadata that corresponds to that step, and violate the security goals. To protect the integrity of software supply chain from such an event, we would like to introduce the integration of TPM.

\subsection{Trusted Platform Module}
Trusted Platform Module (TPM) is an international standard for secure cryptoprocessor, a dedicated microcontroller designed to secure hardware through integrated cryptographic keys. Its standard was conceived by Trusted Computing Group, a computer industry consortium. The primary scope of TPM is to assure integrity of a platform. In this context, "integrity" means "behave as intended", and a "platform" is any computer device regardless of its operating system. It is to ensure that the boot process starts from a trusted combination of hardware and software, and continues until the operating system has fully booted and applications are running.

The TPM consists of several Platform Configuration Registers (PCRs) is to represent the platform software state, the history of the critical software (and configurations) that have
run on the olatform until the present. The TPM initializes all PCRs at power on, typically to either all zeroes or all ones, as specified by the TPM platform specification. The PCR
value is changed through what the TPM calls an \textit{extend} operation. Cryptographically, it can be represented as follows:

PCR new value = Digest of (PCR old value || data to extend)

In words, it takes the the old PCR value and concatenates some data to be extended. The TPM digests the result of the concatenation and stores the resulting digest as the new PCR value. TPM version 1.2 consisted of 16 such registers. But the new version, that is TPM 2.0 consists of 24 registers, giving higher flexibility to run softwares over it.
